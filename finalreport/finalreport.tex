\documentclass{sig-alternate-10pt}

\usepackage{url, enumitem}

\begin{document}

\title{RaptorQ-based File Transfer Protocol} 
\author{ 
Yilong Li\\
\texttt{yilongl@cs.stanford.edu} \and 
Raejoon Jung\\
\texttt{raejoon@stanford.edu} \and 
Yu Yan\\ 
\texttt{yuyan0@stanford.edu} 
}

\maketitle 

\section{Introduction}

\section{Implementation}

\subsection{Reliable file transfer} 
In Tornado transfer, the reliability guarantee is achieved at the application
level by the use of RaptorQ code. This eliminates the need for retransmission
at the file transfer application level, which greatly simplifies the design and
implementation of a reliable file transfer protocol. In Tornado transfer, the
sender has no need to keep track of exactly what symbols are received by the
receiver. Therefore, the ACK message is essentially just a bitmask of 256 bits
(256 is the maximum number of blocks specified in RFC6330) that records which
blocks have been successfully decoded by the receiver. Tornado transfer
protocol has a handshake procedure similar to TCP to establish the connection.
Once the handshake succeeds, the sender starts sending source symbols of each
block in order. Furthermore, in order to compensate for the potential lost
symbols, the sender sends one repair symbol for each previously un-ACK'ed block
after every X source symbols. After all source symbols have been sent, the
sender simply sends repair symbols for each un-ACK'ed block in a round-robin
fashion. Ideally, the repair symbol transmission interval X should be set to a
value such that after all source symbols of block n has been sent, the receiver
has received enough symbols for block n-1 for decoding. This way, the receiver
only needs to keep roughly one block in memory for decoding at a time. The
receiver sends back a heartbeat ACK message constantly to compensate for
potentially lost ACK messages. Besides, it immediately sends back an ACK
message once it decodes a new block to reduce the probability of sender sending
more symbols for the decoded blocks. Once the receiver decodes the entire file,
it simply exits. The sender will also terminate once it figures out that the
receiver exits. This can be done either by relying on the shutdown mechanism of
DCCP socket or through an ICMP destination unreachable message generated by the
receiver.

Parameter setting There are two most important parameters that we can pass on
to the RaptorQ library: symbol size and number of symbols per block.

1. Symbol size In our current implementation, we choose the symbol size to be
1400 bytes to avoid IP fragmentation. We could potentially choose a larger
number to, say, reduce the number of symbols for performance reason described
in the section below. However, the downside of a larger symbol size is that
each symbol may be fragmented at the IP layer and the loss of each fragment
results in the loss of the entire symbol. In other words, the nice property of
digital fountain that every packet received contributes to the decoding of the
entire file is no longer preserved. Currently, we have not quantified the
effect of a larger symbol size.

2. Number of symbols per block The number of symbols per block is critical to
the performance of encoding and decoding. Generally speaking, we would like to
keep it as small as possible. RFC 6330 does not allow us to explicitly set this
value. Instead, we provide a parameter WS, the maximum size of a block that can
be efficiently decoded in the working memory of the receiver, and RFC 6330
describes the procedure for deriving the number of symbols per block based on
it. This parameter derivation algorithm involves lookups into the hardcoded
RaptorQ matrices and is not very straightforward. Therefore, our current
implementation enumerates parameter WS starting from a small number and
increase it by one each time to search for the smallest legal value of the
number of symbols per block. In practice, this search procedure is fast enough
to be hardly noticeable.

Performance bottlenecks Our current implementation of Tornado transfer has two
performance bottlenecks which limits its practicality. We briefly describe the
problems here and leave the solutions as future work.

1. Precomputation The most computational expensive in RaptorQ encoding/decoding
process is the process of precomputing intermediate symbols for each block. The
time complexity of the precomputation is cubic in the number of symbols per
block. We currently use a background thread for precomputing intermediate
symbols while transmitting symbols. However, this has become a bottleneck for
larger file size. For instance, for a file of size 100MB, the smallest number
of symbols per block that is legal with respect to RFC 6330 is 296.

2. Decoding Once intermediate symbols have been precomputed, even though both
encoding and decoding are linear in the number of symbols, decoding tends to
fall behind encoding for two reasons. First, encoding is a stream operation
that takes constant time to generate the next symbol, while decoding is a batch
operation that only happens after enough symbols of a block have been received.
Second, decoding is inherently slower than encoding in the current
implementation of libRaptorQ. To resolve this bottleneck, 

\subsection{Congestion control}
Tornado Transfer utilizes DCCP, the Datagram Congestion Control Protocol, to provide congestion control mechanism for unreliable datagrams. DCCP is designed to make it easy to deploy delay-sensitive applications, such as streaming media, which prefer timeliness to reliability. It is also appropriate for Tornado Transfer because we have offered reliability using RaptorQ in the application layer. 

There are mainly two categories of congestion control algorithms in DCCP--- CCID 2 and CCID 3. CCID 2 denotes TCP-like congestion control that describes Additive Increase Multiplicative Decrease (AIMD) congestion control mechanism including several other features in TCP. CCID 2 is suitable for applications like Tornado Transfer that achieve maximum throughput over long term. CCID 3 denotes TCP-Friendly Rate Control that describes an rate-controlled congestion control mechanism. CCID 3 is suitable for streaming applications due to its lower variation in terms of throughput.

We choose to adopt DCCP as transport layer protocol with CCID 2 enabled for the sake of its simplicity. DCCP has been in the Linux kernel and creating a DCCP socket to send and receive is much like creating a TCP socket. We simply replace UDP with DCCP, without having to add congestion control mechanism on top of UDP by ourselves.

The drawbacks of using DCCP sockets are notable as well. According to our experiments in mahimahi and RFC 5597, NAT (Network Address Translation) support for DCCP is not functioning properly, which results in very little use of DCCP. Besides, its congestion control mechanism, such as replying ACKs, is hidden from us. The black box makes it harder for us to debug when running into problems.

In the future work, we plan to implement congestion control algorithm, e.g. LEDBAT (Low Extra Delay Background Transport), above UDP on our own.


\section{Evaluation}

\section{Contribution}

\section{Reflection}

\section{Conclusion}

\bibliography{proposal} 
\bibliographystyle{acm}

\end{document}
