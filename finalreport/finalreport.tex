\documentclass{sig-alternate-10pt}

\usepackage{url, enumitem}

\begin{document}

\title{RaptorQ-based File Transfer Protocol} 
\author{ 
Yilong Li\\
\texttt{yilongl@cs.stanford.edu} \and 
Raejoon Jung\\
\texttt{raejoon@stanford.edu} \and 
Yu Yan\\ 
\texttt{yuyan0@stanford.edu} 
}

\maketitle 

\section{Introduction}

\section{Implementation}

\subsection{Reliable file transfer} 

\subsection{Congestion control}
Tornado Transfer utilizes DCCP, the Datagram Congestion Control Protocol, to provide congestion control mechanism for unreliable datagrams. DCCP is designed to make it easy to deploy delay-sensitive applications, such as streaming media, which prefer timeliness to reliability. It is also appropriate for Tornado Transfer because we have offered reliability using RaptorQ in the application layer. 

There are mainly two categories of congestion control algorithms in DCCP--- CCID 2 and CCID 3. CCID 2 denotes TCP-like congestion control that describes Additive Increase Multiplicative Decrease (AIMD) congestion control mechanism including several other features in TCP. CCID 2 is suitable for applications like Tornado Transfer that achieve maximum throughput over long term. CCID 3 denotes TCP-Friendly Rate Control that describes an rate-controlled congestion control mechanism. CCID 3 is suitable for streaming applications due to its lower variation in terms of throughput.

We choose to adopt DCCP as transport layer protocol with CCID 2 enabled for the sake of its simplicity. DCCP has been in the Linux kernel and creating a DCCP socket to send and receive is much like creating a TCP socket. We simply replace UDP with DCCP, without having to add congestion control mechanism on top of UDP by ourselves.

The drawbacks of using DCCP sockets are notable as well. According to our experiments in mahimahi and RFC 5597, NAT (Network Address Translation) support for DCCP is not functioning properly, which results in very little use of DCCP. Besides, its congestion control mechanism, such as replying ACKs, is hidden from us. The black box makes it harder for us to debug when running into problems.

In the future work, we plan to implement congestion control algorithm, e.g. LEDBAT (Low Extra Delay Background Transport), above UDP on our own.

\section{Evaluation}

\section{Contribution}

\section{Reflection}

\section{Conclusion}

\bibliography{proposal} 
\bibliographystyle{acm}

\end{document}
